\chapter{Estat de l'art}

\section{Algorismes de recomanació}

Els algorismes de recomanació són algorismes, que, com el seu nom indica, són emprats per a recomanar algúna cosa a algún individu. Aquesta definició, per simple que pugui semblar, introdueix dos elements molt importants. Introdueix el concepte de l'element a ser recomanat, d'ara en endavant el \emph{producte}, i el concepte de qui rep la recomanació, d'ara en endevant, l'\emph{usuari}. Des d'aquest punt, apareixeràn aleshores tres mètodes per atacar el problema d'aconseguir una recomanació de qualitat:

\begin{itemize}
	\item Recomanadors col·laboratius \\
		Aquests recomanadors es basen en tenir un conjunt d'\emph{usuaris} que valorin els \emph{productes} i, amb aquesta informació, determinen usuaris similars entre ells. Aleshores recomenarà a l'\emph{usuari} aquells \emph{productes} que hagin agradat als \emph{usuaris semblants} amb ell. El principal problema que tenen aquests recomanadors és que pateixen molt a l'inici del projecte, ja que cal tenir informació de molts \emph{usuaris} i molts \emph{productes} per tal de proveïr recomanacions encertades.
	\item Recomanadors basats en contingut \\
		Aquest altre sistema el que fa és categoritzar els \emph{productes}. Aleshores, quan l'usuari indica quins \emph{productes} li agraden, sap quines son les característiques més importants per a aquest \emph{usuari}.
	\item Recomanadors híbrids \\
		Aquest sistema consisteix en intentar unir els recomanadors col·laboratius amb els recomanadors basats en contingut. És el sistema més complex, donada la gran quantitat de variables que hi participen, però alhora també és el sistema que pot arribar a donar millors resultats, si es ben implementat i parametritzat.
\end{itemize}

\subsection{Recomanació de pel·lícules}

Existeixen moltes webs que es dediquen a la recomanació de pel·lícules. Per a acotar l'anàlisi de les tècniques que es fan servir reduirem el conjunt a 10.\cite{top-ten-film-recommenders}

Com havíem comentat a la secció anterior, es pot distingir fàcilment entre els \emph{recomanadors col·laboratius} i els \emph{recomanadors basats en contingut}. Exemples del primer conjunt podrien ser, per exemple, Netflix, Movielens, Flixter i Criticker. Per altra banda, formarien part del segon conjunt webs tals com Rotten Tomatoes i Jinni. Finalment, hi ha un tercer conjunt que són els que es basen en rebre com a input una pel·lícula i retornar-ne d'altres. Aquestes webs, com serien IMDb, Clerkdogs o Nanocrowd, molt probablement es basen en un \emph{recomanador basat en contingut} el qual extreu les característiques de la pel·lícula mencionada i en busquen d'altres que tinguin paràmetres similars.

\section{Importància de la recomanació a la web}

La recomanació a la web, que fins fa un temps no era massa comuna, cada cop està guanyant més importància tot i que no en siguem conscients. Un dels grans exemples de recomanació a internet és la publicitat. Un altre gran exemple de recomanacions el trobem als e-commerce, on fins i tot ja hi ha empreses de tercers que es dediquen a implementar els sistemes de recomanació, evitant així al desenvolupador del comerç on-line aquesta feina \cite{brainsins}.

Un exemple que és interessant de remarcar és \emph{Netflix}. Aquesta empresa va veure clar que els sistemes de recomanació eren importants, fins al punt que va organitzar un concurs on, els desenvolupadors que aconseguissin trobar el millor algorisme de recomanació, s'emportarien un premi d'un milió de dòlars \cite{netflix-prize}.

\section{Conjunt de proves}

Per a poder provar els diferents recomanadors que s'utilitzaran cal un conjunt de proves. Donat que, encara que es publiquès la web, aconseguir un conjunt de proves suficientment gran és un procès lent, s'ha utilitzat un conjunt de proves bastant comú en el camp de la recomanació, el de Movielens \cite{movielens-dataset}.

Movielens ha posat a disposició pública tres conjunts de dades diferents, un de cent mil puntuacions, un de un milió de puntuacions, i un de deu milions. El de deu milions té la limitació de que no han donat informació sobre els usuaris mentres que els altres dos conjunts si que en contenen.