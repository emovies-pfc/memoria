\chapter{Estat de l'art}

\section{Algoritmes de recomanació}

Els algoritmes de recomanació són algoritmes, que, com el seu nom indica, són utilitzats per a recomanar algúna cosa a algún individu. Aquesta definició, per simple que pugui semblar, introdueix dos elements molt importants. Introdueix el concepte de l'element a ser recomanat, d'ara en endavant el \emph{producte}, i el concepte de qui rep la recomanació, d'ara en endevant, l'\emph{usuari}.Desde aquest punt, apareixeràn aleshores tres mètodes per a atacar el problema d'aconseguir una recomanació de qualitat:

\begin{itemize}
	\item Recomanadors col·laboratius \\
		Aquests recomanadors es basen en tenir un conjunt d'\emph{usuaris} que valorin els \emph{productes}, i amb aquesta informació determinen usuaris similars entre ells. Aleshores, a l\emph{usuari} li recomanarà \emph{productes} que hagin agradat als \emph{usuaris} semblants amb ell. El principal problema que tenen aquests recomanadors és que pateixen molt a l'inici del projecte, ja que cal tenir informació de molts \emph{usuaris} i molts \emph{productes} per tal de proveïr recomanacions encertades.
	\item Recomanadors basats en contingut \\
		Aquests altre sistema el que fà és categoritzar els \emph{productes}. Aleshores, quan l'usuari indica quins \emph{productes} li agraden, sap quines son les característiques més importants per a aquest \emph{usuari}.
	\item Recomanadors híbrids \\
		Aquest sistema consisteix en intentar unir els recomanadors col·laboratis amb els recomanadors basats en contingut. És el sistema més complex, donada la gran quantitat de variables que hi participen, però a l'hora també és el sistema que pot arribar a donar millors resultats, si es ben implementat i paratemeritzat.
\end{itemize}

A aquest projecte, el 

\subsection{Recomanació de pèl·licules}

Existeixen moltes webs que es dediquen a la recomanació de pèl·licules. Per a acotar l'anàlisi de les tècniques que es fan servir, reduirem el conjunt a \cite{top-ten-film-recommenders}.

Com haviem comentat a la secció anterior, es pot distingir fàcilment entre els \emph{recomanadors col·laboratius} i els \emph{recomanadors basats en contingut}. Exemples del primer conjunt podrien ser, per exemple, Netflix, Movielens, Flixter ¡ Criticker. Per altra banda, formarien part del segon conjunt webs tals com Rotten Tomatoes i Jinni. Finalment, hi ha un tercer conjunt, que són els que es basen en rebre com a input una pèl·licula i retornar-te'n d'altres. Aquestes webs, com serien IMDb, Clerkdogs o Nanocrowd, molt probablement, es basen en un \emph{recomanador basat en contingut}. Aleshores, extreuen les característiques de la pèl·licula mencionada, i en busquen d'altres que tinguin parametres similars.

\section{Xarxes neuronals}

Les xarxes neuronals és una branca de l'\emph{Inteligencia Artificial}, una branca de l'informàtica l'objectiu de la qual és aconseguir que un ordinador sigui capaç d'arribar a conclusions "intel·ligents" d'un problema. Les xarxes neuronals són la branca de l'inteligencia artificial que busca aconseguir-ho mitjançant l'imitació del funcionament del cervell humà.

Per a l'imitació del cervell humà, les xarxes neuronals el que fan és, simplificant molt el model biològic en el que es basen:

\begin{itemize}
	\item Definir un element anomenat neurona, la qual té un estat, activa o inactiva.
	\item Conectar aquestes neurones entre elles, fent que l'estat, i per tant, la sortida d'una neurona, depengui de la suma de les entrades, multiplicades per un pes que es pot donar per a cadascuna de les connexions.
	\item Agrupar les neurones en conjunts lògics anomenats capes. Normalment les neurones d'una capa no estàn connectades entre elles, sinó que estàn connectades únicament amb les neurones de la capa inmediatament superior i inmediatament inferior.
\end{itemize}

Hi ha molts mètodes per a entrenar aquests elements, les xarxes neuronals, per a diferents finalitats. Donada l'extensió d'aquest camp, ens centrarem en aquest projecte només analitzarem un tipus de xarxa neuronal, la qual serà bona per a l'objectiu a aconseguir, que és implementar un recomanador.

\section{Recomanació mitjançant Xarxes Neuronals Artificials}
\label{sec:estat-de-lart-ann}
INCOMPLET

\begin{comment}
El camp de la recomanació mitjançant Xarxes Neuronals Artificials (d'ara en endevant ANN, que que són les sigles per a Artificial Neural Network) és un tema que ha estat tractat en multitut de tesis.

Es pot apreciar que és factible el desenvolupament d'una ANN mitjançant xarxes neuronals. És pot desenvolupar aquest sistema de recomanació tant basant-se en contingut (Content Based Filtering) com basant-se en les opinions dels demés usuaris d'una aplicació (Collaborative Filtering).

El principal problema de les xarxes neuronals és que tenen un temps d'aprenentatge molt llarg\cite{faster-ann-recomender}. Hi ha un estudi\cite{collaborative-filtering-som-cbr} que, el que fà, és utilitzar una ANN per tal de categoritzar els patrons dels usuaris. Aquest proces és dut com un proces \emph{off-line}, que no necesita ser executat cada cop que un usuari solicita una recomanació. Mitjançant alguns altres algoritmes d'aprenentatge, acaba donant recomanacions de forma considerablement ràpida.

Com es pot veure, hi ha hagut bastanta investigació en aquest camp, el que fa que disposem de bastanta documentació per a desenvolupar el projecte.
\end{comment}