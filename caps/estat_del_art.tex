\chapter{Estat de l'art}

\section{Algoritmes de recomanació}

Els algoritmes de recomanació són algoritmes, que, com el seu nom indica, són utilitzats per a recomanar algúna cosa a algún individu. Aquesta definició, per simple que pugui semblar, introdueix dos elements molt importants. Introdueix el concepte de l'element a ser recomanat, d'ara en endavant el \emph{producte}, i el concepte de qui rep la recomanació, d'ara en endevant, l'\emph{usuari}.Desde aquest punt, apareixeràn aleshores tres mètodes per a atacar el problema d'aconseguir una recomanació de qualitat:

\begin{itemize}
	\item Recomanadors col·laboratius \\
		Aquests recomanadors es basen en tenir un conjunt d'\emph{usuaris} que valorin els \emph{productes}, i amb aquesta informació determinen usuaris similars entre ells. Aleshores, a l\emph{usuari} li recomanarà \emph{productes} que hagin agradat als \emph{usuaris} semblants amb ell. El principal problema que tenen aquests recomanadors és que pateixen molt a l'inici del projecte, ja que cal tenir informació de molts \emph{usuaris} i molts \emph{productes} per tal de proveïr recomanacions encertades.
	\item Recomanadors basats en contingut \\
		Aquests altre sistema el que fà és categoritzar els \emph{productes}. Aleshores, quan l'usuari indica quins \emph{productes} li agraden, sap quines son les característiques més importants per a aquest \emph{usuari}.
	\item Recomanadors híbrids \\
		Aquest sistema consisteix en intentar unir els recomanadors col·laboratis amb els recomanadors basats en contingut. És el sistema més complex, donada la gran quantitat de variables que hi participen, però a l'hora també és el sistema que pot arribar a donar millors resultats, si es ben implementat i paratemeritzat.
\end{itemize}

\subsection{Recomanació de pèl·licules}

Existeixen moltes webs que es dediquen a la recomanació de pèl·licules. Per a acotar l'anàlisi de les tècniques que es fan servir, reduirem el conjunt a 10.\cite{top-ten-film-recommenders}

Com haviem comentat a la secció anterior, es pot distingir fàcilment entre els \emph{recomanadors col·laboratius} i els \emph{recomanadors basats en contingut}. Exemples del primer conjunt podrien ser, per exemple, Netflix, Movielens, Flixter ¡ Criticker. Per altra banda, formarien part del segon conjunt webs tals com Rotten Tomatoes i Jinni. Finalment, hi ha un tercer conjunt, que són els que es basen en rebre com a input una pèl·licula i retornar-te'n d'altres. Aquestes webs, com serien IMDb, Clerkdogs o Nanocrowd, molt probablement, es basen en un \emph{recomanador basat en contingut}. Aleshores, extreuen les característiques de la pèl·licula mencionada, i en busquen d'altres que tinguin parametres similars.
