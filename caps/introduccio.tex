\chapter{Introducció}
\section{Descripció del problema}

Actualment, a qualsevol pàgina web, hi ha massa informació per a que l'usuari pugui consumir-la tota. En cas de buscar alguna cosa concreta, l'usuari utilitzarà algun sistema de búsqueda per tal de trobar el que necesita. Però hi ha un gran nombre d'usuaris que no busquen res en concret, sinó que busquen informació que els hi resulti rellevant. Per a aquest segon grup d'usuaris, s'utilitzen sistemes recomanadors, que es el que s'ha analitzat en aquest projecte.

\section{Motivació}

La motivació darrera d'aquest projecte es assolir uns majors coneixements sobre els sistemes recomanadors. A més a més, interesa veure quins sistemes són aplicables al mon web, que requereix de un grau de dinamisme important.

\section{Objectiu}

L'objectiu d'aquest projecte seria la creacció d'una xarxa social amb temàtica orientada a les pèl·licules. El primer objectiu d'aquest aplicatiu web, el qual es busca dur a terme durant el desenvolupament d'aquest Projecte de Final de Carrera, seria, doncs implementar aquest recomanador.

Aleshores, es podria separar l'objectiu d'aquest projecte, en dues parts:

\begin{itemize}
\item Desenvolupament de l'interficie web: S'haurà de desenvolupar tota una interficie web amb la que l'usuari pugui interactuar. Aquesta requerirà d'unes parts que seràn explicades més en detall en un capitol de la memòria final, que serien un un sistema per a la inserció i edició de pèl·licules, un altre sistema per a mostrar la informació sobre la pèl·licula i finalment algun sistema per a evaluar-lo.
\item Sistema recomanador: El sistema recomanador, investigant diferents alternatives a utilitzar.
\end{itemize}

\section{Planificació}

PENDENT