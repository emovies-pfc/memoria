\chapter{Introducció}
\section{Descripció del problema}

Actualment, a qualsevol pàgina web, hi ha massa informació per a que l'usuari pugui consumir-la tota. En cas de buscar alguna cosa concreta, l'usuari utilitzarà algun sistema de cerca per tal de trobar el que necesita. Però hi ha un gran nombre d'usuaris que no busquen res específic, sinó que busquen informació que els hi resulti rellevant. Per a aquest segon grup d'usuaris, s'utilitzen sistemes recomanadors, que es el que s'ha analitzat en aquest projecte.

\section{Motivació}

La motivació darrera d'aquest projecte es assolir uns majors coneixements sobre els sistemes recomanadors. A més a més, interessa veure quins sistemes són aplicables al món web, el qual requereix d'un grau de dinamisme important.

\section{Objectiu}

L'objectiu d'aquest projecte seria la creació d'una xarxa social amb temàtica orientada a les pel·lícules. El primer objectiu d'aquesta aplicació web seria, doncs, implementar aquest recomanador.

Aleshores es podria separar l'objectiu d'aquest projecte en dues parts:

\begin{itemize}
\item Desenvolupament de l'interficie web: S'haurà de desenvolupar tota una interficie web amb la que l'usuari pugui interactuar. Aquesta requerirà d'unes parts que seràn explicades més en detall en un capítol de la memòria final, que serien un sistema per a l'inserció i edició de pel·lícules, un altre sistema per a mostrar la informació sobre la pel·lícula i finalment algun sistema per a avaluar-lo.
\item Sistema recomanador: El sistema recomanador, investigant diferents alternatives a utilitzar.
\end{itemize}

\section{Planificació temporal}

La planificació es pot veure a la taula \ref{table-planing}. D'alguns punts es realitzarà una petita explicació.

\begin{table}[h!]
  \caption{Planificació temporal}
  \label{table-planing}
\begin{tabular}{|c|c|}
	\hline
	\emph{Tasca} & \emph{Durada} \\ \hline
	Anàlisi de l'estat de l'art & 3 setmanes  \\ \hline
	Desenvolupament de modificacions sobre recomanadors trobats & 2 setmanes  \\ \hline
	Devenvolupament de la plataforma web & 1 mes  \\ \hline
	Integració de la plataforma web amb el sistema recomanador & 1 setmana  \\ \hline
	Obtenció de resultats dels recomanadors & 2 setmanes  \\ \hline
	Redacció de la memòria & 3 setmanes \\
	\hline
\end{tabular}
\end{table}

\subsection{Anàlisi de l'estat de l'art}

S'ha de buscar informació sobre els diferents sistemes recomanadors que hi ha per tal de realitzar una mica d'exploració abans de començar a mirar implementacions concretes.

\subsection{Desenvolupament de les modificacions als recomanadors}

Tot i que hi ha moltes implementacions de recomanadors, hi ha algunes que no s'adapten exactament al que es vol, per tant s'han de realitzar modificacions.

\subsection{Obtenció de resultats dels recomanadors}

S'han d'obtindre resultats dels diferents recomanadors utilitzats per tal de poder comparar-los de forma objectiva.

\section{Estructura de la memòria}

A l'inici d'aquesta memòria es pot veure l'estat de l'art, que introdueix una mica els diferents conceptes tractats dins d'aquesta. Seguidament venen els dos principals components del projecte i, per tant, de la memòria. Primerament, hi ha l'apartat de l'implementació web, on s'explica el que s'ha dut a terme per tal de realitzar el frontal web. A continuació, s'explica tota la feina que s'ha realitzat respecte als recomanadors. Per a finalitzar, l'últim capítol parla de les conclusions, explicant també possibles camins per seguir amb el treball que s'ha realitzat en aquest projecte.