\chapter{Planificació temporal}

En aquesta secció s'explicarà la planificació temporal que procuraré dur a terme. Es pot observar que, la data de lliurament de la memòria sigui el 14 de juny, aquesta planificació no engloba fins a aquesta data. Aquest marge de temps, d'\textbf{un mes}, es donat per a evitar possibles imprevistos. En cas de que aquests no es donin, s'anirà extenent el projecte per a cobrir aquest temps adicional.

\section{Implementació de la plataforma web}

La primera implementació de l'aplicatiu web, que consitiria en un senzill CRUD\footnote{De l'anglés, es diu d'un sistema que permet \emph{crear}, \emph{llegir}, \emph{modificar} i \emph{eliminar} (Create, Read, Update, Delete).} per a pèl·licules, usuaris, i puntuacions dels usuaris per a aquestes pèl·licules. A més, s'haurà d'importar a l'aplicatiu dades d'alguna base de dades externa, en aquest projecte, l'utilitzada serà la de movielens\cite{movielens-dataset}. Per a aquesta tasca el temps a utilitzar serà d'unes \textbf{tres setmanes}.

A més de l'implementació, s'haurà d'afegir un mínim de diseny per a la web. Donat que aquest no és l'enfoc del projecte, es realitzarà un diseny bastant bàsic, per al qual es disposarà d'\textbf{una setmana}.

\section{Implementació del sistema recomanador}

L'implementació del sistema recomanador serà la part més complexa. A més, aquest recomanador sempre es possible anar millorant-lo. Es disposarà d'\textbf{un mes} per a la primera implementació usable del recomanador, que seria el primer a integrar amb la web. Després d'això, es disposarà d'unes \textbf{tres setmanes} adicionals per a millorar aquest recomanador.

\section{Integració de la plataforma web amb el sistema recomanador}

L'integració entre la plataforma web i el recomanador no hauria de ser excesivament complexa, ja que unicament caldrà integrar dos mètodes, el d'entrenament de la xarxa neural, i el d'obtenir recomanacions per a un usuari/per a un conjunt de pèl·licules. A aquesta tasca s'hi assignaràn \textbf{dues setmanes}.

\section{Redacció de la memòria}

Per a la redacció de la memoria, suposant que durant el transcurs dels anteriors processos, ja s'hi anirà treballant, s'hi assignarà, entre redacció i correccions, un temps de \textbf{tres setmanes}.
