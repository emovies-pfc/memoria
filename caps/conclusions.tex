\chapter{Conclusions}

\section{Desenvolupament web}

Sobre el desenvolupament de la pàgina web, s'ha desenvolupat una web que és molt senzilla, però que compleix amb l'objectiu, que era donar un frontal per a que l'usuari pugui puntuar pel·lícules, i obtenir recomanacions.

\section{Recomanador}

S'han provat 3 implementacions diferents de recomanadors. S'ha pogut veure que dues d'aquestes (SVD i Slope-One) s'han desmarcat fàcilment del recomanador basat en veïnatge, que es el que s'havia donat al curs, i per tant el que més coneixia i en el que més s'ha treballat.

\section{Possibles ampliacions}

Entre les moltes possibles ampliacions a aquest treball, hi ha dues que mereixen especial menció.

\subsection{Avaluació d'altres paradigmes de recomanació}

Únicament s'ha avaluat un dels paradigmes de la recomanació, que és el filtratge col·laboratiu, per tant encara queda la branca dels recomanadors basats en contingut i dels recomanadors híbrids, que poden donar bastant de joc.

\subsection{Avaluació d'altres recomanadors col·laboratius}

S'han analitzat 3 algoritmes recomanadors, els més coneguts, que son el basat en veïnatge, el Slope-One i el SVD. Es podrien analitzar altres, com per exemple recomanadors basats en xarxes neurals (FALTA CITA) i recomanadors que tinguin en compte quan s'ha realitzat la puntuació (FALTA CITA).